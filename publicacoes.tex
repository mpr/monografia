\chapter{Lista de Publicações}
\label{chap:publicacoes}

\newcommand{\enquote}[1]{``#1''}

Nesta seção, apresentamos os artigos, em ordem cronológica, publicados em congressos e revistas nacionais e internacionais relacionados ao desenvolvimento desta tese. Cada artigo é acompanhado de breve descrição.

\section{Artigos Publicados em Revista}

\begin{enumerate}

\item \MakeUppercase{ Riu, I., Torres, R. C., Seixas, J. M., \MakeLowercase{et~al.}},
  \enquote{Integration of the Trigger and Data Acquisition Systems in {ATLAS}},
  {\em IEEE Transactions on Nuclear Science\/}, v.~55, n.~1, pp.~106--112,
  fevereiro 2008.

Durante o ano de 2006 e 2007 foram obtidos diversos progressos no que tange a integração do TDAQ com os detetores do ATLAS. Foi desenvolvida uma configuração para  TDAQ, já contendo aproximadamente 80 processadores alocados para o modelo final. Vários ensaios técnicos foram elaborados envolvendo toda a cadeia de filtragem do TDAQ, e aspectos críticos para o sistema final, como tempos de processamento, foram estudados usando diferentes algoritmos de filtragem, bem como diferentes módulos do sistema de fluxo de dados.

\item \MakeUppercase{ Sim{\~o}es, J.~A., Dobson, M., Torres, R. C.,
  \MakeLowercase{et~al.}}, \enquote{The {ATLAS} {DAQ} System Online
  Configurations Database Service Challenge}, {\em Journal of Physics:
  Conference Series\/}, v.~119, n. 022004, pp.~1--11, 2008.

Apresenta os requisitos necessários para os serviços de configuração do sistema de filtragem do ATLAS. A proposta desenvolvida para este serviço é discutida e comparada com métodos baseados em base de dados relacionais. Os resultados apresentados justificam a escolha da metodologia proposta para prover o serviço de configuração do TDAQ.

\item \MakeUppercase{ Torres, R.~C., Seixas, J.~M., {dos Anjos}, A.,
  \MakeLowercase{et~al.}}, \enquote{Online Electron/Jet Neural High-level
  Trigger Over Independent Calorimetry Information}, {\em Proceedings of
  Science\/}, , n. PoS(ACAT)039, pp.~1--15, 2008.

O processo de discriminação elétron / jato no segundo n;ivel de filtragem do TDAQ é abordado. Neste trabalho, cada evento de calorimetria é pré-processado topologicamente usando soma em anéis, e, em seguida, as componentes independentes dos mesmos são extraídas e utilizadas para treinar uma rede neural artificial. Os resultados obtidos mostram que o modelo proposto atinge níveis de eficiência superiores ao modelo atualmente em uso no segundo nível de filtragem do TDAQ.


\item \MakeUppercase{ {{dos} Anjos}, A., Torres, R.~C., Seixas, J.~M.,
  \MakeLowercase{et~al.}}, \enquote{Neural Triggering System Operating on High
  Resolution Calorimetry Information}, maio 2006.

Apresenta o sistema de discriminação neural para operação no segundo nível de filtragem do TDAQ usando informaçòes de calorimetria. Para a redução da dimensão dos eventos, o método topológico de soma em anéis é adotado. Os anéis gerados alimentam uma rede neural artificial, encarregada do processo de decisão final. O modelo proposto atingiu 97\% de eficiência de deteção de elétrons para um falso alarme de apenas 3\%. O processo total de decisão gasta apenas 500 $\mu$s, tornando-o adequado para operação \emph{online}.

\end{enumerate} 


\section{Artigos Publicados em Anais de Congressos}

\begin{enumerate}

\item \MakeUppercase{ Torres, R.~C., Seixas, J.~M.}, \enquote{Configura{\c c}{\~a}o
  Autom{\'a}tica para um Sistema de Filtragem Online de Alta Complexidade},
  {\em Congresso Brasileiro de Autom{\'a}tica ({CBA})\/}, pp.~1--6, 2008, Juiz de Fora, Minas Gerais.

Neste artigo, o mecanismo automatizado para a geração de base de dados de configuração da infraestrutura do TDAQ é  apresentado. Um modelo provendo a automatização da configuração de toda a cadeia de filtragem é apresentado, e os resultados mostram que a proposta encontra-se amplamente difundida na comunidade do TDAQ, sendo amplamente usada em testes de larga escala.

\item \MakeUppercase{ Burckhart-Chromek, D., Torres, R. C., {dos Anjos}, A.,
  \MakeLowercase{et~al.}}, \enquote{Testing on a Large Scale: Running the
  {ATLAS} Data Acquisition and High Level Trigger Software on 700 {PC} Nodes}.
\newblock In: {\em Proceedings of the Computing in High Energy and Nuclear
  Physics (CHEP)\/}, pp. 1--6, 2006, Mumbai, Índia.
  
  Este paper apresenta os resultados obtidos com os testes em larga escala do sistema de filtragem do ATLAS, com um máximo de 700 nós de processamento. Os resultados dos testes permitiram que inúmeras melhorias do TDAQ pudessem ser implementadas, além da identificação de áreas críticas, cuja visualização só é possível com a utilizaçào de um grande número de nós de processamento. 

\item \MakeUppercase{ {dos Anjos}, A., Torres, R.~C., Ferreira, B.~C.,
  \MakeLowercase{et~al.}}, \enquote{Discrimina{\c c}{\~a}o Neural de
  El{\'e}trons no Segundo N{\'\i}vel do Trigger do ATLAS}, {\em Encontro
  Nacional de F{\'\i}sica de Part{\'\i}culas e Campos\/}, pp.~1--5, 2005, São Lourenço, Minas Gerais.

\item \MakeUppercase{ {dos Anjos}, A., Torres, R.~C., Ferreira, B.~C.,
  \MakeLowercase{et~al.}}, \enquote{Otimiza{\c c}{\~a}o do Sistema de Trigger
  do Segundo N{\'\i}vel do {ATLAS} Baseado em Calorimetria}, {\em Encontro
  Nacional de F{\'\i}sica de Part{\'\i}culas e Campos\/}, pp.~1--8, 2005, São Lourenço, Minas Gerais.

\end{enumerate}


\section{Resumos Publicados em Anais de Congressos}

\begin{enumerate}

\item \MakeUppercase{ Torres, R.~C., Seixas, J.~M.}, \enquote{Ferramenta de
  Configura{\c c}{\~a}o para o Sistema de Filtragem do {ATLAS}}, {\em Encontro
  Nacional de F{\'\i}sica de Part{\'\i}culas e Campos\/}, 2007, Águas de Lindóia, São Paulo.
  
  É apresentado o estado atual da o mecanismo de configuração automática da infraestrutura do sistema de filtragem de alto nível do ATLAS. Novas possibilidades de operação foram introduzidas, como o suporte a múltiplos servidores RDB e IS.

\item \MakeUppercase{ Torres, R.~C., Seixas, J.~M.}, \enquote{Ferramenta de
  Configura{\c c}{\~a}o para o Sistema de Filtragem do {ATLAS}}, {\em Encontro
  Nacional de F{\'\i}sica de Part{\'\i}culas e Campos\/}, 2006, Águas de Lindóia, São Paulo.

  É apresentada a primeira abordagem para a automatização da configuração da infraestrutura do TDAQ. Nesta abordagem, já é possível configurar toda a cadeia de filtragem em alto nível, gerando grande facilidade de operação do sistema de filtragem.


\end{enumerate}